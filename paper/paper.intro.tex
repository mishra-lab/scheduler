Hospital departments provide services where patient needs, and thus the system's demands, often exceed the available supply. In particular, on-call schedules for a fixed number of health-care providers are central to the efficient running of hospitals. Carefully allocated schedules are meant to simultaneously ensure sufficient resources are provided to patients while not overworking clinicians to prevent costly mistakes [ref]. It is common practice for on-call schedules in hospitals to be created manually, yet manually-created schedules are prone to errors and potential for various biases [ref]. First, when there is a large number of clinicians in a single department, or the constraints that need to be satisfied by the department are very complex, a manual method may not provide an optimal schedule. Second, such methods are likely to overlook certain constraints that must be maintained to have an operational department, such as preventing many consecutive work blocks from being assigned or ensuring clinicians are allocated a specific amount of work blocks throughout the year. Third, manual scheduling is often time-consuming for the person developing the schedule. For these reasons, it is important to develop automated methods that can generate optimal schedules that satisfy the given constraints of the hospital department. \\
%For example, it is important that a hospital department allocates its resources, such as the availability of a finite number of clinicians, optimally, to ensure the best possible service for its patients.

Automated methods to optimize schedules have been studied and applied in many industries, including transportation \cite{aickelin_improved_2006, goel_truck_2012, gunther_combined_2010}, manufacturing \cite{al-yakoob_mixed-integer_2007, al-yakoob_column_2008, alfares_simulation_2007}, retail \cite{chapados_retail_2011, nissen_automatic_2010} and military \cite{horn_scheduling_2007, laguna_modeling_2005}. Of special interest to a clinician on-call scheduling problem are the approaches to schedule nurses, who often work in shifts. In the nurse scheduling problem, the goal is to find an optimal assignment of nurses to shifts that satisfies all of the hard constraints, such as hospital regulations, and as many soft constraints as possible, which may include nurse preferences. A wide variety of approaches, including exact and heuristic approaches, have been used to solve the nurse scheduling problem: integer linear programming \cite{azaiez_0-1_2005, trilling_nurse_2006, widyastiti_nurses_2016}, network flows \cite{el_adoly_new_2018}, genetic algorithms \cite{aickelin_exploiting_2000, jan_evolutionary_2000, kawanaka_genetic_2001}, simulated annealing \cite{jaszkiewicz_metaheuristic_1997}, and artificial intelligence \cite{abdennadher_nurse_nodate, li_hybrid_2003}. A comprehensive literature review of these and other methods applied to nurse rostering is presented in \cite{burke_state_2004}. \\

%An extensive literature review of these and other methods is presented by [??]. We will briefly summarize the main ideas of some of these approaches. \\ SM - don't need to introduce that you will do this for this type of paper I think - would do in a thesis chapter though.

Many of the approaches to nurse scheduling were designed to satisfy the requirements of a specific hospital department which causes a large number of variables and constraints to be incorporated into the problem formulation. While these department-specific approaches allow end-users to find precise schedules that satisfy the needs of the department and the preferences of the nurses and clinicians in that department, they are difficult to readily adapt to other departments in the same hospital or other hospitals. % explain why hard to adapt?... what makes their generalizablility/adaptability limited?
Moreover, the large number of variables and constraints also leads to computational complexity issues \cite{goos_complexity_1996}, especially when using exact methods for finding the solution. In this paper, we tackle a version of the nurse scheduling problem arising from a case study of one clinical division, providing two different services simultaneously (general infectious disease (ID) consults; and HIV consults service) at St. Michael's Hospital in Toronto, Canada. Our goal is to (1) present a simple integer linear programming formulations for the scheduling problem as developed for the hospital, and describe the adaptability of the formulation to solving similar problems in other departments; (2) compare the performance of the ILP scheduler to the results of the manual approach; and (3) analyze the robustness of the algorithm in difficult instances of the problem. \\
% present a simple formulation for the problem developed and tested at the hospital after switching from a manual approach to scheduling; and (2) analyze the performance of integer linear programming in solving difficult instances of the problem and compare the results with those of the manual approach; and (3) describe the adaptability of the formulation as a basic framework for solving similar problems in other departments. \\ %make #3 an objective

We begin by describing the problem in Section \ref{sec:problem}, and presenting our ILP formulation in Section \ref{sec:methods}. Next, we compare the results of our formulation to manually-created schedules, and evaluate the performance of the algorithm on simulated data in Section \ref{sec:results}. Finally, we discuss and interpret the results in Section \ref{sec:discussion}. % list the main contents/elements of the paper.