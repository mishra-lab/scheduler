In this section we discuss the lessons learned from developing the scheduling tool,
as well as the implications of our approach as deployed in the hospital.
The development of the scheduling solution started from a relaxed version of the problem
(without constraints on consecutive blocks/weekends).
We first investigated a maximum flow formulation for solving the problem.
In this formulation, the problem is represented as a graph $G=(V, E)$ where
the vertices comprise clinicians and blocks/weekends, and the edges comprise
potential assignments of clinician to a block/weekend.
A similar approach using maximum flow has been used in~\cite{brucker_personnel_2011, smet_polynomially_2016, el_adoly_new_2018, moz_integer_2003}.
The advantages of a maximum flow formulation are in the computational efficiency
of the algorithm and the interpretability of the constraints and solution in terms of the graph [???].
However, a max flow formulation makes it difficult to incorporate new constraints.
For instance, in order to accommodate the NCB and NCW constraints,
additional edges and nodes need to be added to the graph, potentially reducing 
the computational efficiency and interpretability.
As a result, we opted to reformulate the problem as an ILP, in which the additional constraints
can be incorporated more easily.

The ILP formulation went through several stages of design and tweaking.
For instance, we originally aimed to optimize only the BR and WR constraints.
However, after receiving feedback on some of the test schedules, we learned that 
clinicians would prefer to have some of their weekend assignments closer to the block assignments.
Once we were satisfied with the schedules produced by the algorithm, we targeted the
user interface. Our goal was to develop an easy-to-use interface for the administrative staff to use,
without requiring knowledge of the algorithm. [DL - do we need to elaborate more on this?]

Since the development of the tool, it has been used at St. Michael's Hospital
to generate schedules for the years 2019-21.
The tool is used by administrative staff in the division in order to generate a potential schedule
for the ID and HIV departments. 
The staff provides the following data as input to the scheduler:
(1) maximum and minimum required number of blocks for each clinician, for each division;
(2) a set of time-off requests for each clinician;
(3) a set of holiday dates for the current calendar year.
The supplied time-off requests in (2) are mapped to an appropriate block time-off request by the scheduler.
For example, if a clinician requests time-off during weeks 2 and 3 of the year, the scheduler
will interpret the request as time-off during block 1 (covering weeks 1-2) and 2 (covering weeks 3-4).
The scheduler allows the user to disable certain constraints when generating a schedule
in order to allow a broader range of solutions and prevent issues with infeasibility. 
It also allows the user to generate multiple schedules by randomly shuffling the input list of clinicians.
Once a candidate schedule has been created, the department head may manually
adjust the schedule to better accommodate some of the time-off requests
and any additional requirements
(although at the cost of breaking some of the hard constraints).
Table \ref{tbl:2019-schedule-comparison} presents a comparison of the generated schedule and the final schedule after manual adjustments
for the year 2019. We see that the manual adjustments included splitting or merging of certain blocks
and local swaps between clinicians.
Additionally, blocks and weekends may be adjusted manually due to newly hired clinicians or fellows being 
assigned to cover some of the assignments.
For the year 2020, the generated schedules provided a good starting point for the schedule.
However, due to the COVID-19 pandemic and an increase in the demand and workload of ID clinicians in the department,
the schedules had to be adjusted manually as new requirements emerged.
This is something that could be improved in future iterations of the model
-- specifically, the ability to (re-)generate shorter term schedules which are appropriately
tuned based on a partial completed schedule provided as input.
Overall, the generated schedule from the ILP model provide a good global template
to which local adjustments can be easily performed post-hoc.

\begin{table}[htbp]
  \centering
  \caption{Comparison of the automatically generated (ILP solution) schedule and the final schedule after manual adjustments for 2019.}%
  \label{tbl:2019-schedule-comparison}%
	\begin{adjustbox}{scale=0.8}
	\begin{tabular}{c||ccc||ccc}
		\multicolumn{1}{c||}{\multirow{2}[1]{*}{Week \#}} &                                                                                      \multicolumn{3}{c||}{ILP Solution}                                                                                      &                                                                                    \multicolumn{3}{c}{Manual Adjustment}                                                                                     \\
		                                                  &                                HIV                                 &                                 ID                                 &                              Weekend                               &                                HIV                                 &                                 ID                                 &                              Weekend                               \\ \midrule\midrule
		                        1                         &               \cellcolor[rgb]{ .788,  .788,  .788}D                &                  \cellcolor[rgb]{ 1,  .851,  .4}C                  &                  \cellcolor[rgb]{ 1,  .851,  .4}C                  &               \cellcolor[rgb]{ .788,  .788,  .788}D                &                \cellcolor[rgb]{ .957,  .69,  .518}E                &                \cellcolor[rgb]{ .957,  .69,  .518}E                \\
		                        2                         &               \cellcolor[rgb]{ .788,  .788,  .788}D                &                  \cellcolor[rgb]{ 1,  .851,  .4}C                  &               \cellcolor[rgb]{ .788,  .788,  .788}D                &               \cellcolor[rgb]{ .788,  .788,  .788}D                &                \cellcolor[rgb]{ .957,  .69,  .518}E                &               \cellcolor[rgb]{ .557,  .663,  .859}F                \\
		                        3                         &                   \cellcolor[rgb]{ .8,  .2,  0}L                   & \cellcolor[rgb]{ .251,  .251,  .251}\textcolor[rgb]{ 1,  1,  1}{I} &                   \cellcolor[rgb]{ .8,  .2,  0}L                   &               \cellcolor[rgb]{ .788,  .788,  .788}D                &                   \cellcolor[rgb]{ .8,  .2,  0}L                   &               \cellcolor[rgb]{ .788,  .788,  .788}D                \\
		                        4                         &                   \cellcolor[rgb]{ .8,  .2,  0}L                   & \cellcolor[rgb]{ .251,  .251,  .251}\textcolor[rgb]{ 1,  1,  1}{I} &               \cellcolor[rgb]{ .557,  .663,  .859}F                &               \cellcolor[rgb]{ .788,  .788,  .788}D                &                   \cellcolor[rgb]{ .8,  .2,  0}L                   &                   \cellcolor[rgb]{ .6,  .8,  0}K                   \\
		                        5                         &                  \cellcolor[rgb]{ 1,  .851,  .4}C                  &                \cellcolor[rgb]{ .957,  .69,  .518}E                &                \cellcolor[rgb]{ .957,  .69,  .518}E                &                  \cellcolor[rgb]{ 1,  .851,  .4}C                  & \cellcolor[rgb]{ .251,  .251,  .251}\textcolor[rgb]{ 1,  1,  1}{I} &               \cellcolor[rgb]{ .459,  .443,  .443}H                \\
		                        6                         &                  \cellcolor[rgb]{ 1,  .851,  .4}C                  &                \cellcolor[rgb]{ .957,  .69,  .518}E                &                \cellcolor[rgb]{ .518,  .592,  .69}G                &                  \cellcolor[rgb]{ 1,  .851,  .4}C                  & \cellcolor[rgb]{ .251,  .251,  .251}\textcolor[rgb]{ 1,  1,  1}{I} &                \cellcolor[rgb]{ .518,  .592,  .69}G                \\
		                        7                         &               \cellcolor[rgb]{ .788,  .788,  .788}D                &                   \cellcolor[rgb]{ .8,  .2,  0}L                   &               \cellcolor[rgb]{ .788,  .788,  .788}D                &               \cellcolor[rgb]{ .788,  .788,  .788}D                &                \cellcolor[rgb]{ .957,  .69,  .518}E                &                   \cellcolor[rgb]{ .8,  .2,  0}L                   \\
		                        8                         &               \cellcolor[rgb]{ .788,  .788,  .788}D                &                   \cellcolor[rgb]{ .8,  .2,  0}L                   &               \cellcolor[rgb]{ .557,  .663,  .859}F                &               \cellcolor[rgb]{ .788,  .788,  .788}D                &                \cellcolor[rgb]{ .957,  .69,  .518}E                &                   \cellcolor[rgb]{ .8,  .2,  0}L                   \\
		                        9                         & \cellcolor[rgb]{ .251,  .251,  .251}\textcolor[rgb]{ 1,  1,  1}{I} &                   \cellcolor[rgb]{ .6,  .8,  0}K                   &                   \cellcolor[rgb]{ .6,  .8,  0}K                   &                  \cellcolor[rgb]{ 1,  .851,  .4}C                  & \cellcolor[rgb]{ .251,  .251,  .251}\textcolor[rgb]{ 1,  1,  1}{I} &               \cellcolor[rgb]{ .557,  .663,  .859}F                \\
		                       10                         & \cellcolor[rgb]{ .251,  .251,  .251}\textcolor[rgb]{ 1,  1,  1}{I} &                   \cellcolor[rgb]{ .6,  .8,  0}K                   & \cellcolor[rgb]{ .251,  .251,  .251}\textcolor[rgb]{ 1,  1,  1}{I} &                  \cellcolor[rgb]{ 1,  .851,  .4}C                  & \cellcolor[rgb]{ .251,  .251,  .251}\textcolor[rgb]{ 1,  1,  1}{I} & \cellcolor[rgb]{ .251,  .251,  .251}\textcolor[rgb]{ 1,  1,  1}{I} \\
		                       11                         &               \cellcolor[rgb]{ .788,  .788,  .788}D                &                  \cellcolor[rgb]{ 1,  .851,  .4}C                  &                  \cellcolor[rgb]{ 1,  .851,  .4}C                  &               \cellcolor[rgb]{ .788,  .788,  .788}D                &                   \cellcolor[rgb]{ .6,  .8,  0}K                   &               \cellcolor[rgb]{ .788,  .788,  .788}D                \\
		                       12                         &               \cellcolor[rgb]{ .788,  .788,  .788}D                &                  \cellcolor[rgb]{ 1,  .851,  .4}C                  &                \cellcolor[rgb]{ .518,  .592,  .69}G                &               \cellcolor[rgb]{ .788,  .788,  .788}D                &                   \cellcolor[rgb]{ .6,  .8,  0}K                   &                \cellcolor[rgb]{ .518,  .592,  .69}G                \\
		                       13                         &                   \cellcolor[rgb]{ .8,  .2,  0}L                   &                   \cellcolor[rgb]{ .6,  .8,  0}K                   &                   \cellcolor[rgb]{ .6,  .8,  0}K                   &                  \cellcolor[rgb]{ 1,  .851,  .4}C                  &                \cellcolor[rgb]{ .518,  .592,  .69}G                &               \cellcolor[rgb]{ .459,  .443,  .443}H                \\
		                       14                         &                   \cellcolor[rgb]{ .8,  .2,  0}L                   &                   \cellcolor[rgb]{ .6,  .8,  0}K                   &                \cellcolor[rgb]{ .957,  .69,  .518}E                &                  \cellcolor[rgb]{ 1,  .851,  .4}C                  &                \cellcolor[rgb]{ .518,  .592,  .69}G                &                \cellcolor[rgb]{ .957,  .69,  .518}E                \\
		                       15                         &                  \cellcolor[rgb]{ 1,  .851,  .4}C                  &               \cellcolor[rgb]{ .557,  .663,  .859}F                &                  \cellcolor[rgb]{ 1,  .851,  .4}C                  &                   \cellcolor[rgb]{ .8,  .2,  0}L                   &                   \cellcolor[rgb]{ .8,  .2,  0}L                   &                   \cellcolor[rgb]{ .6,  .8,  0}K                   \\
		                       16                         &                  \cellcolor[rgb]{ 1,  .851,  .4}C                  &               \cellcolor[rgb]{ .557,  .663,  .859}F                &               \cellcolor[rgb]{ .557,  .663,  .859}F                &                   \cellcolor[rgb]{ .8,  .2,  0}L                   &                   \cellcolor[rgb]{ .8,  .2,  0}L                   &                   \cellcolor[rgb]{ 1,  .6,  .8}J                   \\
		                       17                         & \cellcolor[rgb]{ .251,  .251,  .251}\textcolor[rgb]{ 1,  1,  1}{I} &                   \cellcolor[rgb]{ .8,  .2,  0}L                   &                   \cellcolor[rgb]{ .8,  .2,  0}L                   & \cellcolor[rgb]{ .251,  .251,  .251}\textcolor[rgb]{ 1,  1,  1}{I} &                  \cellcolor[rgb]{ 1,  .851,  .4}C                  &                   \cellcolor[rgb]{ .8,  .2,  0}L                   \\
		                       18                         & \cellcolor[rgb]{ .251,  .251,  .251}\textcolor[rgb]{ 1,  1,  1}{I} &                   \cellcolor[rgb]{ .8,  .2,  0}L                   & \cellcolor[rgb]{ .251,  .251,  .251}\textcolor[rgb]{ 1,  1,  1}{I} & \cellcolor[rgb]{ .251,  .251,  .251}\textcolor[rgb]{ 1,  1,  1}{I} &                  \cellcolor[rgb]{ 1,  .851,  .4}C                  &                \cellcolor[rgb]{ .957,  .69,  .518}E                \\
		                       19                         &               \cellcolor[rgb]{ .788,  .788,  .788}D                &               \cellcolor[rgb]{ .459,  .443,  .443}H                &               \cellcolor[rgb]{ .788,  .788,  .788}D                &               \cellcolor[rgb]{ .788,  .788,  .788}D                &                   \cellcolor[rgb]{ .6,  .8,  0}K                   & \cellcolor[rgb]{ .251,  .251,  .251}\textcolor[rgb]{ 1,  1,  1}{I} \\
		                       20                         &               \cellcolor[rgb]{ .788,  .788,  .788}D                &               \cellcolor[rgb]{ .459,  .443,  .443}H                &               \cellcolor[rgb]{ .459,  .443,  .443}H                &               \cellcolor[rgb]{ .788,  .788,  .788}D                &                   \cellcolor[rgb]{ .6,  .8,  0}K                   &               \cellcolor[rgb]{ .788,  .788,  .788}D                \\
		                       21                         &                  \cellcolor[rgb]{ 1,  .851,  .4}C                  &               \cellcolor[rgb]{ .557,  .663,  .859}F                &               \cellcolor[rgb]{ .557,  .663,  .859}F                & \cellcolor[rgb]{ .251,  .251,  .251}\textcolor[rgb]{ 1,  1,  1}{I} &               \cellcolor[rgb]{ .557,  .663,  .859}F                &                \cellcolor[rgb]{ .518,  .592,  .69}G                \\
		                       22                         &                  \cellcolor[rgb]{ 1,  .851,  .4}C                  &               \cellcolor[rgb]{ .557,  .663,  .859}F                &                \cellcolor[rgb]{ .957,  .69,  .518}E                & \cellcolor[rgb]{ .251,  .251,  .251}\textcolor[rgb]{ 1,  1,  1}{I} &               \cellcolor[rgb]{ .557,  .663,  .859}F                &                \cellcolor[rgb]{ .957,  .69,  .518}E                \\
		                       23                         &               \cellcolor[rgb]{ .788,  .788,  .788}D                & \cellcolor[rgb]{ .251,  .251,  .251}\textcolor[rgb]{ 1,  1,  1}{I} & \cellcolor[rgb]{ .251,  .251,  .251}\textcolor[rgb]{ 1,  1,  1}{I} &               \cellcolor[rgb]{ .788,  .788,  .788}D                &                  \cellcolor[rgb]{ 1,  .851,  .4}C                  & \cellcolor[rgb]{ .251,  .251,  .251}\textcolor[rgb]{ 1,  1,  1}{I} \\
		                       24                         &               \cellcolor[rgb]{ .788,  .788,  .788}D                & \cellcolor[rgb]{ .251,  .251,  .251}\textcolor[rgb]{ 1,  1,  1}{I} &               \cellcolor[rgb]{ .459,  .443,  .443}H                &               \cellcolor[rgb]{ .788,  .788,  .788}D                &                  \cellcolor[rgb]{ 1,  .851,  .4}C                  &               \cellcolor[rgb]{ .788,  .788,  .788}D                \\
		                       25                         &                  \cellcolor[rgb]{ 1,  .851,  .4}C                  &                \cellcolor[rgb]{ .957,  .69,  .518}E                &                  \cellcolor[rgb]{ 1,  .851,  .4}C                  & \cellcolor[rgb]{ .251,  .251,  .251}\textcolor[rgb]{ 1,  1,  1}{I} &               \cellcolor[rgb]{ .557,  .663,  .859}F                &               \cellcolor[rgb]{ .459,  .443,  .443}H                \\
		                       26                         &                  \cellcolor[rgb]{ 1,  .851,  .4}C                  &                \cellcolor[rgb]{ .957,  .69,  .518}E                &                \cellcolor[rgb]{ .957,  .69,  .518}E                & \cellcolor[rgb]{ .251,  .251,  .251}\textcolor[rgb]{ 1,  1,  1}{I} &               \cellcolor[rgb]{ .557,  .663,  .859}F                &                \cellcolor[rgb]{ .518,  .592,  .69}G                \\
		                       27                         & \cellcolor[rgb]{ .251,  .251,  .251}\textcolor[rgb]{ 1,  1,  1}{I} &                \cellcolor[rgb]{ .518,  .592,  .69}G                &                \cellcolor[rgb]{ .518,  .592,  .69}G                &               \cellcolor[rgb]{ .788,  .788,  .788}D                &                \cellcolor[rgb]{ .957,  .69,  .518}E                &                \cellcolor[rgb]{ .957,  .69,  .518}E                \\
		                       28                         & \cellcolor[rgb]{ .251,  .251,  .251}\textcolor[rgb]{ 1,  1,  1}{I} &                \cellcolor[rgb]{ .518,  .592,  .69}G                & \cellcolor[rgb]{ .251,  .251,  .251}\textcolor[rgb]{ 1,  1,  1}{I} &               \cellcolor[rgb]{ .788,  .788,  .788}D                &                \cellcolor[rgb]{ .957,  .69,  .518}E                &               \cellcolor[rgb]{ .557,  .663,  .859}F                \\
		                       29                         &               \cellcolor[rgb]{ .788,  .788,  .788}D                &               \cellcolor[rgb]{ .459,  .443,  .443}H                &               \cellcolor[rgb]{ .459,  .443,  .443}H                &                  \cellcolor[rgb]{ 1,  .851,  .4}C                  &               \cellcolor[rgb]{ .459,  .443,  .443}H                &                   \cellcolor[rgb]{ .6,  .8,  0}K                   \\
		                       30                         &               \cellcolor[rgb]{ .788,  .788,  .788}D                &               \cellcolor[rgb]{ .459,  .443,  .443}H                &                   \cellcolor[rgb]{ .6,  .8,  0}K                   &                  \cellcolor[rgb]{ 1,  .851,  .4}C                  &               \cellcolor[rgb]{ .459,  .443,  .443}H                &                   \cellcolor[rgb]{ 1,  .6,  .8}J                   \\
		                       31                         &                  \cellcolor[rgb]{ 1,  .851,  .4}C                  &                   \cellcolor[rgb]{ .8,  .2,  0}L                   &                   \cellcolor[rgb]{ .8,  .2,  0}L                   &               \cellcolor[rgb]{ .788,  .788,  .788}D                &                   \cellcolor[rgb]{ .6,  .8,  0}K                   &               \cellcolor[rgb]{ .459,  .443,  .443}H                \\
		                       32                         &                  \cellcolor[rgb]{ 1,  .851,  .4}C                  &                   \cellcolor[rgb]{ .8,  .2,  0}L                   &                \cellcolor[rgb]{ .518,  .592,  .69}G                &               \cellcolor[rgb]{ .788,  .788,  .788}D                &                   \cellcolor[rgb]{ .6,  .8,  0}K                   &               \cellcolor[rgb]{ .788,  .788,  .788}D                \\
		                       33                         & \cellcolor[rgb]{ .251,  .251,  .251}\textcolor[rgb]{ 1,  1,  1}{I} &                   \cellcolor[rgb]{ .6,  .8,  0}K                   & \cellcolor[rgb]{ .251,  .251,  .251}\textcolor[rgb]{ 1,  1,  1}{I} &                  \cellcolor[rgb]{ 1,  .851,  .4}C                  &                  \cellcolor[rgb]{ 1,  .851,  .4}C                  & \cellcolor[rgb]{ .251,  .251,  .251}\textcolor[rgb]{ 1,  1,  1}{I} \\
		                       34                         & \cellcolor[rgb]{ .251,  .251,  .251}\textcolor[rgb]{ 1,  1,  1}{I} &                   \cellcolor[rgb]{ .6,  .8,  0}K                   &               \cellcolor[rgb]{ .557,  .663,  .859}F                &                   \cellcolor[rgb]{ .8,  .2,  0}L                   &                  \cellcolor[rgb]{ 1,  .851,  .4}C                  &                   \cellcolor[rgb]{ .6,  .8,  0}K                   \\
		                       35                         &               \cellcolor[rgb]{ .788,  .788,  .788}D                &                  \cellcolor[rgb]{ 1,  .851,  .4}C                  &                  \cellcolor[rgb]{ 1,  .851,  .4}C                  & \cellcolor[rgb]{ .251,  .251,  .251}\textcolor[rgb]{ 1,  1,  1}{I} &               \cellcolor[rgb]{ .459,  .443,  .443}H                & \cellcolor[rgb]{ .251,  .251,  .251}\textcolor[rgb]{ 1,  1,  1}{I} \\
		                       36                         &               \cellcolor[rgb]{ .788,  .788,  .788}D                &                  \cellcolor[rgb]{ 1,  .851,  .4}C                  &               \cellcolor[rgb]{ .788,  .788,  .788}D                &               \cellcolor[rgb]{ .788,  .788,  .788}D                &               \cellcolor[rgb]{ .459,  .443,  .443}H                &               \cellcolor[rgb]{ .557,  .663,  .859}F                \\
		                       37                         &                   \cellcolor[rgb]{ .8,  .2,  0}L                   &                \cellcolor[rgb]{ .518,  .592,  .69}G                &                \cellcolor[rgb]{ .518,  .592,  .69}G                &                   \cellcolor[rgb]{ .8,  .2,  0}L                   &                   \cellcolor[rgb]{ .6,  .8,  0}K                   &               \cellcolor[rgb]{ .788,  .788,  .788}D                \\
		                       38                         &                   \cellcolor[rgb]{ .8,  .2,  0}L                   &                \cellcolor[rgb]{ .518,  .592,  .69}G                &                   \cellcolor[rgb]{ .6,  .8,  0}K                   & \cellcolor[rgb]{ .251,  .251,  .251}\textcolor[rgb]{ 1,  1,  1}{I} &                   \cellcolor[rgb]{ .6,  .8,  0}K                   &               \cellcolor[rgb]{ .459,  .443,  .443}H                \\
		                       39                         &               \cellcolor[rgb]{ .788,  .788,  .788}D                &                \cellcolor[rgb]{ .957,  .69,  .518}E                &                \cellcolor[rgb]{ .957,  .69,  .518}E                &                   \cellcolor[rgb]{ .8,  .2,  0}L                   &                \cellcolor[rgb]{ .518,  .592,  .69}G                &                \cellcolor[rgb]{ .518,  .592,  .69}G                \\
		                       40                         &               \cellcolor[rgb]{ .788,  .788,  .788}D                &                \cellcolor[rgb]{ .957,  .69,  .518}E                &                   \cellcolor[rgb]{ .8,  .2,  0}L                   &                   \cellcolor[rgb]{ .8,  .2,  0}L                   &                \cellcolor[rgb]{ .518,  .592,  .69}G                & \cellcolor[rgb]{ .251,  .251,  .251}\textcolor[rgb]{ 1,  1,  1}{I} \\
		                       41                         &                  \cellcolor[rgb]{ 1,  .851,  .4}C                  &                \cellcolor[rgb]{ .518,  .592,  .69}G                &                \cellcolor[rgb]{ .518,  .592,  .69}G                &                   \cellcolor[rgb]{ .8,  .2,  0}L                   &               \cellcolor[rgb]{ .459,  .443,  .443}H                &                   \cellcolor[rgb]{ .6,  .8,  0}K                   \\
		                       42                         &                  \cellcolor[rgb]{ 1,  .851,  .4}C                  &                \cellcolor[rgb]{ .518,  .592,  .69}G                &               \cellcolor[rgb]{ .459,  .443,  .443}H                & \cellcolor[rgb]{ .251,  .251,  .251}\textcolor[rgb]{ 1,  1,  1}{I} &               \cellcolor[rgb]{ .459,  .443,  .443}H                &                \cellcolor[rgb]{ .957,  .69,  .518}E                \\
		                       43                         &                   \cellcolor[rgb]{ .8,  .2,  0}L                   &               \cellcolor[rgb]{ .557,  .663,  .859}F                &               \cellcolor[rgb]{ .557,  .663,  .859}F                &                  \cellcolor[rgb]{ 1,  .851,  .4}C                  &               \cellcolor[rgb]{ .557,  .663,  .859}F                &                \cellcolor[rgb]{ .957,  .69,  .518}E                \\
		                       44                         &                   \cellcolor[rgb]{ .8,  .2,  0}L                   &               \cellcolor[rgb]{ .557,  .663,  .859}F                &               \cellcolor[rgb]{ .788,  .788,  .788}D                & \cellcolor[rgb]{ .251,  .251,  .251}\textcolor[rgb]{ 1,  1,  1}{I} &               \cellcolor[rgb]{ .557,  .663,  .859}F                &               \cellcolor[rgb]{ .557,  .663,  .859}F                \\
		                       45                         &               \cellcolor[rgb]{ .788,  .788,  .788}D                &                   \cellcolor[rgb]{ .6,  .8,  0}K                   &                   \cellcolor[rgb]{ .6,  .8,  0}K                   &               \cellcolor[rgb]{ .788,  .788,  .788}D                &                \cellcolor[rgb]{ .518,  .592,  .69}G                &                \cellcolor[rgb]{ .518,  .592,  .69}G                \\
		                       46                         &               \cellcolor[rgb]{ .788,  .788,  .788}D                &                   \cellcolor[rgb]{ .6,  .8,  0}K                   & \cellcolor[rgb]{ .251,  .251,  .251}\textcolor[rgb]{ 1,  1,  1}{I} &                   \cellcolor[rgb]{ .8,  .2,  0}L                   &                \cellcolor[rgb]{ .518,  .592,  .69}G                &                   \cellcolor[rgb]{ .8,  .2,  0}L                   \\
		                       47                         &                  \cellcolor[rgb]{ 1,  .851,  .4}C                  &                   \cellcolor[rgb]{ .8,  .2,  0}L                   &                   \cellcolor[rgb]{ .8,  .2,  0}L                   &                  \cellcolor[rgb]{ 1,  .851,  .4}C                  &                   \cellcolor[rgb]{ .6,  .8,  0}K                   &               \cellcolor[rgb]{ .788,  .788,  .788}D                \\
		                       48                         &                  \cellcolor[rgb]{ 1,  .851,  .4}C                  &                   \cellcolor[rgb]{ .8,  .2,  0}L                   &               \cellcolor[rgb]{ .788,  .788,  .788}D                &                  \cellcolor[rgb]{ 1,  .851,  .4}C                  &                   \cellcolor[rgb]{ .6,  .8,  0}K                   &               \cellcolor[rgb]{ .459,  .443,  .443}H                \\
		                       49                         &               \cellcolor[rgb]{ .788,  .788,  .788}D                &               \cellcolor[rgb]{ .459,  .443,  .443}H                &               \cellcolor[rgb]{ .459,  .443,  .443}H                &               \cellcolor[rgb]{ .788,  .788,  .788}D                & \cellcolor[rgb]{ .251,  .251,  .251}\textcolor[rgb]{ 1,  1,  1}{I} &                   \cellcolor[rgb]{ .8,  .2,  0}L                   \\
		                       50                         &               \cellcolor[rgb]{ .788,  .788,  .788}D                &               \cellcolor[rgb]{ .459,  .443,  .443}H                &                   \cellcolor[rgb]{ .6,  .8,  0}K                   &               \cellcolor[rgb]{ .788,  .788,  .788}D                &                   \cellcolor[rgb]{ .8,  .2,  0}L                   &                   \cellcolor[rgb]{ .6,  .8,  0}K                   \\
		                       51                         &                  \cellcolor[rgb]{ 1,  .851,  .4}C                  &                   \cellcolor[rgb]{ .6,  .8,  0}K                   &                  \cellcolor[rgb]{ 1,  .851,  .4}C                  &                  \cellcolor[rgb]{ 1,  .851,  .4}C                  &                   \cellcolor[rgb]{ .8,  .2,  0}L                   &               \cellcolor[rgb]{ .459,  .443,  .443}H
	\end{tabular}%
\end{adjustbox}\\[1em]
  \label{tab:addlabel}%
\end{table}%

