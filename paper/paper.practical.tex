In this section we discuss the implications of our approach as deployed in the hospital,
as well as the lessons learned from developing the scheduling tool.
The tool is used by administrative staff in the division in order to generate a potential schedule
for the ID and HIV departments. 
The staff provides the following data as input to the scheduler:
(1) maximum and minimum required number of blocks for each clinician, for each division;
(2) a set of time-off requests for each clinician;
(3) a set of holiday dates for the current calendar year.
The supplied time-off requests in (2) are mapped to an appropriate block time-off request by the scheduler.
For example, if a clinician requests time-off during weeks 2 and 3 of the year, the scheduler
will interpret the request as time-off during block 1 (covering weeks 1-2) and 2 (covering weeks 3-4).
The scheduler allows the user to disable certain constraints when generating a schedule
in order to allow a broader range of solutions and prevent issues with infeasibility. 
It also allows the user to generate multiple schedules by randomly shuffling the input list of clinicians.
Once a candidate schedule has been created, the department head will manually adjust it based on
additional [qualitative-?] constraints to create a final schedule for the division.
Since the development of the tool, it has been used to generate schedules for the years 2019-21.

Figure XX presents a comparison of the generated schedule and the final schedule after manual adjustments
for the year 20XX. We can see...

[TODO-lessons learned]