In this section we discuss the implications of our approach as deployed in the hospital,
as well as the lessons learned from developing the scheduling tool.
The tool is used by administrative staff in the division in order to generate a potential schedule
for the ID and HIV departments. 
The staff provides the following data as input to the scheduler:
(1) maximum and minimum required number of blocks for each clinician, for each division;
(2) a set of time-off requests for each clinician;
(3) a set of holiday dates for the current calendar year.
The supplied time-off requests in (2) are mapped to an appropriate block time-off request by the scheduler.
For example, if a clinician requests time-off during weeks 2 and 3 of the year, the scheduler
will interpret the request as time-off during block 1 (covering weeks 1-2) and 2 (covering weeks 3-4).
The scheduler allows the user to disable certain constraints when generating a schedule
in order to allow a broader range of solutions and prevent issues with infeasibility. 
It also allows the user to generate multiple schedules by randomly shuffling the input list of clinicians.
Once a candidate schedule has been created, the department head will manually adjust it based on
additional [qualitative-?] constraints to create a final schedule for the division.
Since the development of the tool, it has been used to generate schedules for the years 2019-21.
Table XX presents a comparison of the generated schedule and the final schedule after manual adjustments
for the year 2019. We see that the manual adjustments were likely as a result of additional clinicians
and/or fellows being assigned to cover a few of the blocks, as well as certain blocks that were 
split into individual weeks. Overall, the generated schedule provides a good global template,
and local adjustments can be performed post-hoc.

The development of the scheduling solution started from a relaxed version of the problem
(without constraints on consecutive blocks/weekends).
We first investigated a maximum flow formulation for solving the problem.
In this formulation, the problem is represented as a graph $G=(V, E)$ where
the vertices comprise clinicians and blocks/weekends, and the edges comprise
potential assignments of clinician to a block/weekend.
A similar approach using maximum flow has been used in [XX, XX, XX].
The advantages of a maximum flow formulation are in the computational efficiency
of the algorithm and the interpretability of the constraints and solution in terms of the graph [???].
However, a max flow formulation makes it difficult to incorporate new constraints.
For instance, in order to accommodate the NCB and NCW constraints,
additional edges and nodes need to be added to the graph, potentially reducing 
the computational efficiency and interpretability.
As a result, we opted to reformulate the problem as an ILP, in which the additional constraints
can be incorporated more easily.
[...]