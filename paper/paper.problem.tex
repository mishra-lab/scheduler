At St. Michael's Hospital, the division of infectious diseases (ID) offers general ID and HIV consultation on inpatients as two parallel services. Each service provides clinical care throughout the year, during regular work weeks as well as weekends and holidays. The clinicians in the division typically receive a schedule in advance, outlining their on-call service dates for the full year. In the yearly schedule each clinician is assigned to blocks of regular work weeks and weekends. Each block corresponds to two consecutive work weeks. Apart from long (holiday) weekends, a work week starts on Monday at 8 A.M. and ends on Friday at 5 P.M. Accordingly, weekend service starts on Friday at 5 P.M., and ends on Monday at 8 A.M. During the weekend, ID and HIV consultation services are combined and provided by one clinician. During the regular work week the ID and HIV services are led by one clinician each. \\

Several constraints are placed on the on-call assignments. First, each clinician has limits on the number of blocks they can and must work during the year, depending on the type of consultation. For instance, a clinician might have to provide 3-5 blocks of general ID consultation as well as 2-3 blocks of HIV consultation throughout the year. These limits may change from year to year as the number of clinicians in the department changes. Moreover, the schedule should not assign a clinician to work for two consecutive blocks or two consecutive weekends in a row. The schedule should also distribute both regular and holiday weekends equally among all clinicians. \\

In addition to maintaining a balanced work load among clinicians, the schedule should also accommodate their preferences. Clinicians provide their requests for time off ahead of schedule generation so that the requests may be integrated into the schedule. Individuals may specify days, weeks or weekends off, with the understanding that any blocks overlapping with their request will be assigned to a different clinician where possible. For example, if a clinician only requests a given Monday and Tuesday off, the schedule will generally avoid assigning the entire block to that clinician. Clinicians also prefer to have their weekend and block assignments close together, so the schedule should account for this when distributing assignments. A summary of the outlined constraints is given in Table \ref{tbl:constraint-summary}.

\begin{table}[h]
	\centering
	\begin{tabular}{ l l l }
		\hline
		\textbf{Constraint Name} & \textbf{Description}                                                                                                           & \textbf{Type} \\ \hline
		Block Coverage (BC)           & \makecell[l]{each service needs to have exactly \\ one clinician that covers any given block}                                 & Hard          \\ \hline
		Weekend Coverage (WC)        & \makecell[l]{every weekend needs to have exactly \\ one clinician that covers it}                                              & Hard          \\ \hline
		Min/Max (MM)                  & \makecell[l]{for a given service, each clinician can only \\ work between the minimum and maximum \\ number of allowed blocks} & Hard          \\ \hline
		No Consecutive Blocks (NCB)   & \makecell[l]{any clinician should not work \\ two consecutive blocks, across all services}                                    & Hard          \\ \hline
		No Consecutive Weekends (NCW) & \makecell[l]{any clinician should not work two consecutive \\ weekends}                                                        & Hard          \\ \hline
		Equal Weekends (EW)          & \makecell[l]{weekends should be equally distributed \\ between clinicians}                                                     & Hard          \\ \hline
		Equal Holidays (EH)          & \makecell[l]{long weekends should be equally distributed \\ between clinicians}                                                & Hard          \\ \hline
		Block Requests (BR)         & \makecell[l]{each clinician can request to be off service \\ during certain blocks throughout the year}                        & Soft          \\ \hline
		Weekend Requests (WR)        & \makecell[l]{each clinician can request to be off service \\ during certain weekends throughout the year}                      & Soft          \\ \hline
		Block-Weekend Adjacency (BWA)  & \makecell[l]{the block and weekend assignments of a given \\ clinician should be adjacent}                                     & Soft
	\end{tabular}
	\caption{Summary of the constraints for the clinician scheduling problem}
	\label{tbl:constraint-summary}
\end{table}

In most scheduling problems, the constraints can be divided into hard and soft constraints. Hard constraints must be satisfied by any candidate solution, while soft constraints can be used to select a more favourable solution from all candidate solutions. Typically, soft constraints are encoded as objective functions rather than actual constraints. These objective functions are maximized or minimized when finding a solution. In the case of our clinician scheduling problem, we chose Block Requests, Weekend Requests and Block-Weekend Adjacency as our soft constraints, while the rest of the constraints are hard. Though it is important to take clinician requests into account when constructing the schedule, it is crucial that the work-load of the schedule is balanced among all clinicians, and the needs of the patients are fulfilled. It is important to note that - irrespective of how the schedule is generated - clinicians may exchange certain weeks or days throughout the year after the schedule is implemented. The approach to solving the clinician scheduling problem does not account for these future exchanges, and only focuses on the full time horizon.