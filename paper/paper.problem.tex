At St.\ Michael's Hospital, the division of infectious diseases (ID) offers
separate but concurrent services for general ID and HIV consultation. Each
service provides clinical care throughout the year, during regular working hours
and on weekends and holidays. The schedule is created in advance,
outlining all work-week and weekend shifts for the full year.
In the yearly schedule clinicians are assigned to ``blocks'' of
two consecutive regular work weeks and individual weekends. Apart
% JK: when I first read this I thought weekends were also in 2-week blocks
from long (holiday) weekends, a work week starts on Monday at 8 A.M. and ends on
Friday at 5 P.M. Accordingly, weekend service starts on Friday at 5 P.M., and
ends on Monday at 8 A.M. During the weekend, ID and HIV consultation services
are combined and provided by one clinician. During the regular work week the ID
and HIV services are led by one clinician each.
% JK: I thik we need something like:
%     Therefore, our objective is to: assign one clinician to ...
%     while additionally considering the following constraints ...
Therefore, our objective is to assign a single clinician to cover
each service for each block and each weekend of the year, while 
additionally ensuring a balanced workload.

In most scheduling problems, the constraints can be divided into hard and soft
constraints. Hard constraints must be satisfied by any candidate solution, while
soft constraints can be used to select a more favourable solution from a set of
candidate solutions. Typically, soft constraints are therefore encoded as objective
functions which are
maximized when finding a solution. In the case of the clinician
% JK: the direction of max / minimization is arbitrary,
%     I think sufficient to pick one here to avoid confusion.
scheduling problem, we chose departmental regulations and workload balance as hard constraints,
while clinician preference and requests serve as soft constraints.
%we chose Block Requests, Weekend Requests and Block-Weekend
%Adjacency as our soft constraints, while the rest of the constraints are hard.
Although it is important to take clinician requests into account when constructing
the schedule, it is crucial that the workload of the schedule is balanced among
all clinicians, and all services are assigned.
After the schedule is generated, clinicians may
exchange certain weeks or days throughout the year, to fulfill any missed requests.
%Our approach to schedule generation does not
%consider such future exchanges, and instead generates on one full year at a time.
% JK: I think this para (4) should go instead as the 2nd in this section.
%     you could introduce the ideas of hard and soft constraints (this para) first,
%     and then have the current paras (2) and (3) after this one, where the reader
%     is already aware that (2) is hard constraints and (3) is soft.
%     Oh, I see that they're not currently split up exactly, but I think it would be
%     helpful to do so. Maybe double check with SM though.

The following are the departmental and workload constraints placed on the clinician assignments. 
% JK: I think you need to introduce BC and WC constraints here,
%     since they appear in Table 1 without text in the body.
First, each clinician
must work between a minimum and maximum number of blocks of each service during the year.
For instance, one clinician might have to
provide 3-5 blocks of general ID service and 2-3 blocks of HIV service
% JK: trying to reduce synonyms "service" / "consultation"
throughout the year. These limits may be different for each clinician,
and they may change from year to year as
the number of clinicians in the department changes. Second, the schedule
should not assign a clinician to work for two consecutive blocks or two
consecutive weekends. The schedule should also distribute regular
weekends and holiday weekends each equally among all clinicians.

In addition to balancing the workload among clinicians, the schedule
should accommodate their preferences. Clinicians provide their requests for
time off before schedule generation so that the requests may be integrated
into the schedule. Clinicians may specify days, weeks or weekends off, with the
understanding that any blocks overlapping with their request will be assigned to
a different clinician where possible. For example, if a clinician only requests
a given Monday and Tuesday off, the schedule should avoid assigning the
entire block to that clinician. Clinicians also typically prefer to have their weekend and
block assignments side by side, so the schedule should accommodate this where possible.
A summary of the outlined constraints is given in
Table~\ref{tbl:constraint-summary}.

\begin{table}[h]
	\centering
	\caption{Summary of the constraints for the clinician scheduling problem}%
  \label{tbl:constraint-summary}
  % JK: table caption should go at the top.
  %     The spacing can be funny so I use:
  %     \usepackage{caption}\captionsetup[table]{skip=1em}
  %     also labels should be on their own line.
  % JK: below: I would consider making the 2-3 letter abbreviations its own column
	\begin{tabular}{ l c l l }
		\toprule
		% JK: different line weights support table structure
		\textbf{Constraint Name} & \textbf{Abbreviation} & \textbf{Description}                                                                                                                 & \textbf{Type} \\ \midrule
		Block Coverage                                                                  & BC                    & \makecell[l]{each service needs to have
			exactly \\ one clinician that covers any given block}                                     & Hard          \\ \hline
		Weekend Coverage                                                                & WC                    & \makecell[l]{every weekend needs to have
			exactly \\ one clinician that covers it}                                                 & Hard          \\ \hline
		Min/Max                                                                         & MM                    & \makecell[l]{for a given service, each
			clinician can only \\ work between the minimum and maximum \\ number of allowed
			blocks} & Hard          \\ \hline
		No Consecutive Blocks                                                           & NCB                   & \makecell[l]{any clinician should not work \\
			two consecutive blocks, across all services}                                        & Hard          \\ \hline
		No Consecutive Weekends                                                         & NCW                   & \makecell[l]{any clinician should not work two
			consecutive \\ weekends}                                                           & Hard          \\ \hline
		Equal Weekends                                                                  & EW                    & \makecell[l]{weekends should be equally
			distributed \\ between clinicians}                                                        & Hard          \\ \hline
		Equal Holidays                                                                  & EH                    & \makecell[l]{long weekends should be equally
			distributed \\ between clinicians}                                                   & Hard          \\ \hline
		Block Requests                                                                  & BR                    & \makecell[l]{each clinician can request to be
			off service \\ during certain blocks throughout the year}                           & Soft          \\ \hline
		Weekend Requests                                                                & WR                    & \makecell[l]{each clinician can request to be
			off service \\ during certain weekends throughout the year}                         & Soft          \\ \hline
		Block-Weekend Adjacency                                                         & BWA                   & \makecell[l]{the block and weekend
			assignments of a given \\ clinician should be adjacent}                                        & Soft          \\ \bottomrule
		% JK: I think usually table bottom also has a rule                              &                       &                                                                                                                                      &
	\end{tabular}
  % JK: please define Hard & Soft constraints in a footnote
\footnotesize\raggedright
Hard constraints must be satisfied by any candidate schedule. 
Soft constraints are optionally satisfied, and are used to rank the set of candidate solutions.
\end{table}