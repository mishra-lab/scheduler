\subsection{Software}
We developed a software package with a user interface that implements the above LP problem and allows configuration of clinicians at [ref \ref{???}], to be used by the ID division at St. Michael's Hospital. The software was used to generate the results in the following sections, using real data as well as simulated data as input.

\subsection{Infectious Diseases Division}
We used clinician time-off requests and minimum/maximum requirements from 2015-2018 as input data for the LP problem. Tables \ref{tbl:2018-schedule-comparison}, \ref{??}, \ref{??} present the optimal schedule generated using the software as well as the manually created schedule, color-coded to distinguish between the different clinicians assigned.

% Table generated by Excel2LaTeX from sheet '2018'
\begin{table}[h]
%	\tiny
 	\centering
 	\begin{adjustbox}{scale=0.8}
	    \begin{tabular}{c||ccc||ccc}
	    	\multicolumn{1}{c||}{\multirow{2}[1]{*}{Week \#}} & \multicolumn{3}{c||}{LP Solution}                                                                                                                                       & \multicolumn{3}{c}{Historical Data}                                                                                                                                     \\
	    	                                                  &                  HIV                  &                                 ID                                 &                              Weekend                               &                  HIV                  &                                 ID                                 &                              Weekend                               \\ \midrule\midrule
	    	                        1                         & \cellcolor[rgb]{ .663,  .816,  .557}A &                \cellcolor[rgb]{ .957,  .69,  .518}E                &                \cellcolor[rgb]{ .957,  .69,  .518}E                & \cellcolor[rgb]{ .663,  .816,  .557}A &                \cellcolor[rgb]{ .957,  .69,  .518}E                &               \cellcolor[rgb]{ .459,  .443,  .443}H                \\
	    	                        2                         & \cellcolor[rgb]{ .663,  .816,  .557}A &                \cellcolor[rgb]{ .957,  .69,  .518}E                &                \cellcolor[rgb]{ .518,  .592,  .69}G                & \cellcolor[rgb]{ .663,  .816,  .557}A &                \cellcolor[rgb]{ .957,  .69,  .518}E                &               \cellcolor[rgb]{ .663,  .816,  .557}A                \\
	    	                        3                         & \cellcolor[rgb]{ .608,  .761,  .902}B &               \cellcolor[rgb]{ .557,  .663,  .859}F                &               \cellcolor[rgb]{ .557,  .663,  .859}F                & \cellcolor[rgb]{ .608,  .761,  .902}B &               \cellcolor[rgb]{ .459,  .443,  .443}H                &                \cellcolor[rgb]{ .518,  .592,  .69}G                \\
	    	                        4                         & \cellcolor[rgb]{ .608,  .761,  .902}B &               \cellcolor[rgb]{ .557,  .663,  .859}F                &               \cellcolor[rgb]{ .459,  .443,  .443}H                & \cellcolor[rgb]{ .608,  .761,  .902}B &               \cellcolor[rgb]{ .459,  .443,  .443}H                & \cellcolor[rgb]{ .251,  .251,  .251}\textcolor[rgb]{ 1,  1,  1}{I} \\
	    	                        5                         & \cellcolor[rgb]{ .663,  .816,  .557}A &                \cellcolor[rgb]{ .518,  .592,  .69}G                &               \cellcolor[rgb]{ .663,  .816,  .557}A                & \cellcolor[rgb]{ .663,  .816,  .557}A &                \cellcolor[rgb]{ .518,  .592,  .69}G                &               \cellcolor[rgb]{ .557,  .663,  .859}F                \\
	    	                        6                         & \cellcolor[rgb]{ .663,  .816,  .557}A &                \cellcolor[rgb]{ .518,  .592,  .69}G                &                \cellcolor[rgb]{ .957,  .69,  .518}E                & \cellcolor[rgb]{ .663,  .816,  .557}A &                \cellcolor[rgb]{ .518,  .592,  .69}G                &                  \cellcolor[rgb]{ 1,  .851,  .4}C                  \\
	    	                        7                         &   \cellcolor[rgb]{ 1,  .851,  .4}C    &               \cellcolor[rgb]{ .608,  .761,  .902}B                &                  \cellcolor[rgb]{ 1,  .851,  .4}C                  & \cellcolor[rgb]{ .663,  .816,  .557}A &               \cellcolor[rgb]{ .557,  .663,  .859}F                &               \cellcolor[rgb]{ .608,  .761,  .902}B                \\
	    	                        8                         &   \cellcolor[rgb]{ 1,  .851,  .4}C    &               \cellcolor[rgb]{ .608,  .761,  .902}B                &                \cellcolor[rgb]{ .518,  .592,  .69}G                & \cellcolor[rgb]{ .788,  .788,  .788}D &                  \cellcolor[rgb]{ 1,  .851,  .4}C                  &                \cellcolor[rgb]{ .518,  .592,  .69}G                \\
	    	                        9                         & \cellcolor[rgb]{ .788,  .788,  .788}D &               \cellcolor[rgb]{ .459,  .443,  .443}H                &               \cellcolor[rgb]{ .788,  .788,  .788}D                & \cellcolor[rgb]{ .608,  .761,  .902}B &                  \cellcolor[rgb]{ 1,  .851,  .4}C                  &               \cellcolor[rgb]{ .788,  .788,  .788}D                \\
	    	                       10                         & \cellcolor[rgb]{ .788,  .788,  .788}D &               \cellcolor[rgb]{ .459,  .443,  .443}H                &               \cellcolor[rgb]{ .459,  .443,  .443}H                & \cellcolor[rgb]{ .608,  .761,  .902}B &               \cellcolor[rgb]{ .788,  .788,  .788}D                &               \cellcolor[rgb]{ .459,  .443,  .443}H                \\
	    	                       11                         & \cellcolor[rgb]{ .663,  .816,  .557}A & \cellcolor[rgb]{ .251,  .251,  .251}\textcolor[rgb]{ 1,  1,  1}{I} & \cellcolor[rgb]{ .251,  .251,  .251}\textcolor[rgb]{ 1,  1,  1}{I} & \cellcolor[rgb]{ .663,  .816,  .557}A &               \cellcolor[rgb]{ .608,  .761,  .902}B                &               \cellcolor[rgb]{ .557,  .663,  .859}F                \\
	    	                       12                         & \cellcolor[rgb]{ .663,  .816,  .557}A & \cellcolor[rgb]{ .251,  .251,  .251}\textcolor[rgb]{ 1,  1,  1}{I} &               \cellcolor[rgb]{ .608,  .761,  .902}B                & \cellcolor[rgb]{ .663,  .816,  .557}A &               \cellcolor[rgb]{ .608,  .761,  .902}B                &               \cellcolor[rgb]{ .663,  .816,  .557}A                \\
	    	                       13                         & \cellcolor[rgb]{ .608,  .761,  .902}B &               \cellcolor[rgb]{ .557,  .663,  .859}F                &               \cellcolor[rgb]{ .557,  .663,  .859}F                &   \cellcolor[rgb]{ 1,  .851,  .4}C    &               \cellcolor[rgb]{ .459,  .443,  .443}H                &               \cellcolor[rgb]{ .459,  .443,  .443}H                \\
	    	                       14                         & \cellcolor[rgb]{ .608,  .761,  .902}B &               \cellcolor[rgb]{ .557,  .663,  .859}F                &               \cellcolor[rgb]{ .459,  .443,  .443}H                &   \cellcolor[rgb]{ 1,  .851,  .4}C    &               \cellcolor[rgb]{ .459,  .443,  .443}H                & \cellcolor[rgb]{ .251,  .251,  .251}\textcolor[rgb]{ 1,  1,  1}{I} \\
	    	                       15                         &   \cellcolor[rgb]{ 1,  .851,  .4}C    & \cellcolor[rgb]{ .251,  .251,  .251}\textcolor[rgb]{ 1,  1,  1}{I} &                  \cellcolor[rgb]{ 1,  .851,  .4}C                  & \cellcolor[rgb]{ .608,  .761,  .902}B & \cellcolor[rgb]{ .251,  .251,  .251}\textcolor[rgb]{ 1,  1,  1}{I} &                  \cellcolor[rgb]{ 1,  .851,  .4}C                  \\
	    	                       16                         &   \cellcolor[rgb]{ 1,  .851,  .4}C    & \cellcolor[rgb]{ .251,  .251,  .251}\textcolor[rgb]{ 1,  1,  1}{I} &               \cellcolor[rgb]{ .663,  .816,  .557}A                & \cellcolor[rgb]{ .608,  .761,  .902}B & \cellcolor[rgb]{ .251,  .251,  .251}\textcolor[rgb]{ 1,  1,  1}{I} &                \cellcolor[rgb]{ .957,  .69,  .518}E                \\
	    	                       17                         & \cellcolor[rgb]{ .608,  .761,  .902}B &               \cellcolor[rgb]{ .788,  .788,  .788}D                &               \cellcolor[rgb]{ .788,  .788,  .788}D                & \cellcolor[rgb]{ .663,  .816,  .557}A &                \cellcolor[rgb]{ .957,  .69,  .518}E                &               \cellcolor[rgb]{ .788,  .788,  .788}D                \\
	    	                       18                         & \cellcolor[rgb]{ .608,  .761,  .902}B &               \cellcolor[rgb]{ .788,  .788,  .788}D                &               \cellcolor[rgb]{ .557,  .663,  .859}F                & \cellcolor[rgb]{ .663,  .816,  .557}A &                \cellcolor[rgb]{ .957,  .69,  .518}E                &                \cellcolor[rgb]{ .957,  .69,  .518}E                \\
	    	                       19                         & \cellcolor[rgb]{ .663,  .816,  .557}A &               \cellcolor[rgb]{ .459,  .443,  .443}H                &               \cellcolor[rgb]{ .459,  .443,  .443}H                & \cellcolor[rgb]{ .663,  .816,  .557}A &                  \cellcolor[rgb]{ 1,  .851,  .4}C                  &               \cellcolor[rgb]{ .557,  .663,  .859}F                \\
	    	                       20                         & \cellcolor[rgb]{ .663,  .816,  .557}A &               \cellcolor[rgb]{ .459,  .443,  .443}H                & \cellcolor[rgb]{ .251,  .251,  .251}\textcolor[rgb]{ 1,  1,  1}{I} & \cellcolor[rgb]{ .663,  .816,  .557}A &                  \cellcolor[rgb]{ 1,  .851,  .4}C                  &                  \cellcolor[rgb]{ 1,  .851,  .4}C                  \\
	    	                       21                         & \cellcolor[rgb]{ .788,  .788,  .788}D &                  \cellcolor[rgb]{ 1,  .851,  .4}C                  &                  \cellcolor[rgb]{ 1,  .851,  .4}C                  & \cellcolor[rgb]{ .608,  .761,  .902}B &                \cellcolor[rgb]{ .518,  .592,  .69}G                &               \cellcolor[rgb]{ .663,  .816,  .557}A                \\
	    	                       22                         & \cellcolor[rgb]{ .788,  .788,  .788}D &                  \cellcolor[rgb]{ 1,  .851,  .4}C                  &                \cellcolor[rgb]{ .518,  .592,  .69}G                & \cellcolor[rgb]{ .608,  .761,  .902}B &                \cellcolor[rgb]{ .518,  .592,  .69}G                &                  \cellcolor[rgb]{ 1,  .851,  .4}C                  \\
	    	                       23                         & \cellcolor[rgb]{ .608,  .761,  .902}B &                \cellcolor[rgb]{ .957,  .69,  .518}E                &                \cellcolor[rgb]{ .957,  .69,  .518}E                &   \cellcolor[rgb]{ 1,  .851,  .4}C    &               \cellcolor[rgb]{ .557,  .663,  .859}F                &               \cellcolor[rgb]{ .788,  .788,  .788}D                \\
	    	                       24                         & \cellcolor[rgb]{ .608,  .761,  .902}B &                \cellcolor[rgb]{ .957,  .69,  .518}E                &               \cellcolor[rgb]{ .557,  .663,  .859}F                &   \cellcolor[rgb]{ 1,  .851,  .4}C    &               \cellcolor[rgb]{ .557,  .663,  .859}F                &                  \cellcolor[rgb]{ 1,  .851,  .4}C                  \\
	    	                       25                         & \cellcolor[rgb]{ .663,  .816,  .557}A &               \cellcolor[rgb]{ .459,  .443,  .443}H                &               \cellcolor[rgb]{ .663,  .816,  .557}A                &   \cellcolor[rgb]{ 1,  .851,  .4}C    &                  \cellcolor[rgb]{ 1,  .851,  .4}C                  &                \cellcolor[rgb]{ .518,  .592,  .69}G                \\
	    	                       26                         & \cellcolor[rgb]{ .663,  .816,  .557}A &               \cellcolor[rgb]{ .459,  .443,  .443}H                &               \cellcolor[rgb]{ .459,  .443,  .443}H                & \cellcolor[rgb]{ .788,  .788,  .788}D & \cellcolor[rgb]{ .251,  .251,  .251}\textcolor[rgb]{ 1,  1,  1}{I} &               \cellcolor[rgb]{ .788,  .788,  .788}D                \\
	    	                       27                         & \cellcolor[rgb]{ .608,  .761,  .902}B &                  \cellcolor[rgb]{ 1,  .851,  .4}C                  &                  \cellcolor[rgb]{ 1,  .851,  .4}C                  & \cellcolor[rgb]{ .663,  .816,  .557}A &               \cellcolor[rgb]{ .608,  .761,  .902}B                &                \cellcolor[rgb]{ .957,  .69,  .518}E                \\
	    	                       28                         & \cellcolor[rgb]{ .608,  .761,  .902}B &                  \cellcolor[rgb]{ 1,  .851,  .4}C                  &                \cellcolor[rgb]{ .957,  .69,  .518}E                & \cellcolor[rgb]{ .663,  .816,  .557}A &               \cellcolor[rgb]{ .608,  .761,  .902}B                & \cellcolor[rgb]{ .251,  .251,  .251}\textcolor[rgb]{ 1,  1,  1}{I} \\
	    	                       29                         & \cellcolor[rgb]{ .663,  .816,  .557}A &                \cellcolor[rgb]{ .518,  .592,  .69}G                &               \cellcolor[rgb]{ .663,  .816,  .557}A                & \cellcolor[rgb]{ .608,  .761,  .902}B &               \cellcolor[rgb]{ .788,  .788,  .788}D                &               \cellcolor[rgb]{ .788,  .788,  .788}D                \\
	    	                       30                         & \cellcolor[rgb]{ .663,  .816,  .557}A &                \cellcolor[rgb]{ .518,  .592,  .69}G                &               \cellcolor[rgb]{ .608,  .761,  .902}B                & \cellcolor[rgb]{ .608,  .761,  .902}B &               \cellcolor[rgb]{ .788,  .788,  .788}D                &               \cellcolor[rgb]{ .663,  .816,  .557}A                \\
	    	                       31                         &   \cellcolor[rgb]{ 1,  .851,  .4}C    &               \cellcolor[rgb]{ .557,  .663,  .859}F                &                  \cellcolor[rgb]{ 1,  .851,  .4}C                  &   \cellcolor[rgb]{ 1,  .851,  .4}C    &               \cellcolor[rgb]{ .557,  .663,  .859}F                &                \cellcolor[rgb]{ .957,  .69,  .518}E                \\
	    	                       32                         &   \cellcolor[rgb]{ 1,  .851,  .4}C    &               \cellcolor[rgb]{ .557,  .663,  .859}F                &                \cellcolor[rgb]{ .957,  .69,  .518}E                &   \cellcolor[rgb]{ 1,  .851,  .4}C    &               \cellcolor[rgb]{ .557,  .663,  .859}F                &               \cellcolor[rgb]{ .557,  .663,  .859}F                \\
	    	                       33                         & \cellcolor[rgb]{ .608,  .761,  .902}B &               \cellcolor[rgb]{ .788,  .788,  .788}D                &               \cellcolor[rgb]{ .788,  .788,  .788}D                & \cellcolor[rgb]{ .608,  .761,  .902}B &               \cellcolor[rgb]{ .557,  .663,  .859}F                & \cellcolor[rgb]{ .251,  .251,  .251}\textcolor[rgb]{ 1,  1,  1}{I} \\
	    	                       34                         & \cellcolor[rgb]{ .608,  .761,  .902}B &               \cellcolor[rgb]{ .788,  .788,  .788}D                &               \cellcolor[rgb]{ .608,  .761,  .902}B                & \cellcolor[rgb]{ .608,  .761,  .902}B & \cellcolor[rgb]{ .251,  .251,  .251}\textcolor[rgb]{ 1,  1,  1}{I} &                  \cellcolor[rgb]{ 1,  .851,  .4}C                  \\
	    	                       35                         & \cellcolor[rgb]{ .663,  .816,  .557}A & \cellcolor[rgb]{ .251,  .251,  .251}\textcolor[rgb]{ 1,  1,  1}{I} & \cellcolor[rgb]{ .251,  .251,  .251}\textcolor[rgb]{ 1,  1,  1}{I} & \cellcolor[rgb]{ .663,  .816,  .557}A &                \cellcolor[rgb]{ .518,  .592,  .69}G                &                \cellcolor[rgb]{ .518,  .592,  .69}G                \\
	    	                       36                         & \cellcolor[rgb]{ .663,  .816,  .557}A & \cellcolor[rgb]{ .251,  .251,  .251}\textcolor[rgb]{ 1,  1,  1}{I} &                \cellcolor[rgb]{ .518,  .592,  .69}G                & \cellcolor[rgb]{ .663,  .816,  .557}A &                \cellcolor[rgb]{ .518,  .592,  .69}G                & \cellcolor[rgb]{ .251,  .251,  .251}\textcolor[rgb]{ 1,  1,  1}{I} \\
	    	                       37                         & \cellcolor[rgb]{ .788,  .788,  .788}D &                \cellcolor[rgb]{ .518,  .592,  .69}G                &               \cellcolor[rgb]{ .788,  .788,  .788}D                & \cellcolor[rgb]{ .788,  .788,  .788}D &                  \cellcolor[rgb]{ 1,  .851,  .4}C                  &               \cellcolor[rgb]{ .663,  .816,  .557}A                \\
	    	                       38                         & \cellcolor[rgb]{ .788,  .788,  .788}D &                \cellcolor[rgb]{ .518,  .592,  .69}G                &               \cellcolor[rgb]{ .557,  .663,  .859}F                & \cellcolor[rgb]{ .788,  .788,  .788}D &               \cellcolor[rgb]{ .788,  .788,  .788}D                &                \cellcolor[rgb]{ .957,  .69,  .518}E                \\
	    	                       39                         & \cellcolor[rgb]{ .663,  .816,  .557}A &                \cellcolor[rgb]{ .957,  .69,  .518}E                &               \cellcolor[rgb]{ .663,  .816,  .557}A                & \cellcolor[rgb]{ .663,  .816,  .557}A &               \cellcolor[rgb]{ .608,  .761,  .902}B                &               \cellcolor[rgb]{ .788,  .788,  .788}D                \\
	    	                       40                         & \cellcolor[rgb]{ .663,  .816,  .557}A &                \cellcolor[rgb]{ .957,  .69,  .518}E                &               \cellcolor[rgb]{ .608,  .761,  .902}B                & \cellcolor[rgb]{ .608,  .761,  .902}B & \cellcolor[rgb]{ .251,  .251,  .251}\textcolor[rgb]{ 1,  1,  1}{I} & \cellcolor[rgb]{ .251,  .251,  .251}\textcolor[rgb]{ 1,  1,  1}{I} \\
	    	                       41                         & \cellcolor[rgb]{ .608,  .761,  .902}B &                \cellcolor[rgb]{ .518,  .592,  .69}G                &                \cellcolor[rgb]{ .518,  .592,  .69}G                & \cellcolor[rgb]{ .608,  .761,  .902}B & \cellcolor[rgb]{ .251,  .251,  .251}\textcolor[rgb]{ 1,  1,  1}{I} &                \cellcolor[rgb]{ .518,  .592,  .69}G                \\
	    	                       42                         & \cellcolor[rgb]{ .608,  .761,  .902}B &                \cellcolor[rgb]{ .518,  .592,  .69}G                &                \cellcolor[rgb]{ .957,  .69,  .518}E                & \cellcolor[rgb]{ .788,  .788,  .788}D &               \cellcolor[rgb]{ .557,  .663,  .859}F                &               \cellcolor[rgb]{ .557,  .663,  .859}F                \\
	    	                       43                         & \cellcolor[rgb]{ .788,  .788,  .788}D &                  \cellcolor[rgb]{ 1,  .851,  .4}C                  &               \cellcolor[rgb]{ .788,  .788,  .788}D                &   \cellcolor[rgb]{ 1,  .851,  .4}C    &               \cellcolor[rgb]{ .557,  .663,  .859}F                &                  \cellcolor[rgb]{ 1,  .851,  .4}C                  \\
	    	                       44                         & \cellcolor[rgb]{ .788,  .788,  .788}D &                  \cellcolor[rgb]{ 1,  .851,  .4}C                  & \cellcolor[rgb]{ .251,  .251,  .251}\textcolor[rgb]{ 1,  1,  1}{I} & \cellcolor[rgb]{ .788,  .788,  .788}D &                \cellcolor[rgb]{ .957,  .69,  .518}E                & \cellcolor[rgb]{ .251,  .251,  .251}\textcolor[rgb]{ 1,  1,  1}{I} \\
	    	                       45                         & \cellcolor[rgb]{ .663,  .816,  .557}A &               \cellcolor[rgb]{ .557,  .663,  .859}F                &               \cellcolor[rgb]{ .557,  .663,  .859}F                & \cellcolor[rgb]{ .788,  .788,  .788}D &                \cellcolor[rgb]{ .957,  .69,  .518}E                &                \cellcolor[rgb]{ .957,  .69,  .518}E                \\
	    	                       46                         & \cellcolor[rgb]{ .663,  .816,  .557}A &               \cellcolor[rgb]{ .557,  .663,  .859}F                &               \cellcolor[rgb]{ .663,  .816,  .557}A                & \cellcolor[rgb]{ .663,  .816,  .557}A &               \cellcolor[rgb]{ .608,  .761,  .902}B                &               \cellcolor[rgb]{ .663,  .816,  .557}A                \\
	    	                       47                         &   \cellcolor[rgb]{ 1,  .851,  .4}C    &               \cellcolor[rgb]{ .608,  .761,  .902}B                &                  \cellcolor[rgb]{ 1,  .851,  .4}C                  & \cellcolor[rgb]{ .663,  .816,  .557}A &               \cellcolor[rgb]{ .608,  .761,  .902}B                &               \cellcolor[rgb]{ .788,  .788,  .788}D                \\
	    	                       48                         &   \cellcolor[rgb]{ 1,  .851,  .4}C    &               \cellcolor[rgb]{ .608,  .761,  .902}B                & \cellcolor[rgb]{ .251,  .251,  .251}\textcolor[rgb]{ 1,  1,  1}{I} & \cellcolor[rgb]{ .663,  .816,  .557}A &               \cellcolor[rgb]{ .788,  .788,  .788}D                &                \cellcolor[rgb]{ .518,  .592,  .69}G                \\
	    	                       49                         & \cellcolor[rgb]{ .663,  .816,  .557}A &               \cellcolor[rgb]{ .788,  .788,  .788}D                &               \cellcolor[rgb]{ .788,  .788,  .788}D                & \cellcolor[rgb]{ .663,  .816,  .557}A &               \cellcolor[rgb]{ .788,  .788,  .788}D                &               \cellcolor[rgb]{ .557,  .663,  .859}F                \\
	    	                       50                         & \cellcolor[rgb]{ .663,  .816,  .557}A &               \cellcolor[rgb]{ .788,  .788,  .788}D                &               \cellcolor[rgb]{ .608,  .761,  .902}B                & \cellcolor[rgb]{ .608,  .761,  .902}B &                \cellcolor[rgb]{ .518,  .592,  .69}G                &                \cellcolor[rgb]{ .518,  .592,  .69}G                \\
	    	                       51                         & \cellcolor[rgb]{ .608,  .761,  .902}B &                \cellcolor[rgb]{ .518,  .592,  .69}G                &                \cellcolor[rgb]{ .518,  .592,  .69}G                & \cellcolor[rgb]{ .608,  .761,  .902}B &                \cellcolor[rgb]{ .518,  .592,  .69}G                &                \cellcolor[rgb]{ .957,  .69,  .518}E
	    \end{tabular}%
	\end{adjustbox}
	\caption{Comparison of schedules for 2018}
	\label{tbl:2018-schedule-comparison}%
\end{table}%


In order to evaluate the generated schedule and compare it with the manually created schedule, we outline the adherence of each schedule to the constraints presented in section \ref{???}. As we can see, the generated schedule was able to satisfy all mandatory constraints, however for the years of [???] it could not find an optimal schedule while ensuring that the block assignments are spread out ([???] constraint). On the other hand, we can see that the manually created schedule was not able to satisfy all mandatory constraints. In particular, we see that it contains consecutive block assignments for the years of [???], [...]. Evaluating the objectives, we see that [...\textit{request conflicts}]. Moreover, the manual schedule does not attempt to align weekend assignments with block assignments, unlike the optimal schedule found by the software.

\subsection{Simulations}