In recent years, there has been a substantial amount of literature published in
the nurse rostering area aiming to organize the types of problems and approaches
commonly
addressed in the real-world. This research serves to highlight the similarities
and differences
among computationally easy and hard instances of nurse rostering, evaluating
which algorithms
are best geared towards which types of problems, and [...]

De Causmaecker and Berghe [ref] outline a categorization
of nurse rostering problems using a similar approach used in scheduling,
based around the personnel environment, work characteristics and optimization
objective. Previously, there was a lack of a central categorization for
nurse rostering problems found in the literature. Their work attempts
to address this, by creating a classification that takes into account
hard and soft constraints related to nurse preferences, consecutive 
assignments, coverage and various flexibilities in terms of shift and
skill types. They go on to describe where recent examples of the nurse rostering
literature can fit in their classification.

% DL - maybe this should be in complexity section?
Vanhoucke and Maenhout [ref] analyze and compare nurse rostering instances
based on complexity indicators arising from problem constraints. They consider
problem size (in terms of number of nurses, shifts and days), preference distribution
of the nurses, coverage distribution and time-related constraints (such as weekend
and consecutive constraints) as the four indicators for problem complexity.
Then, they generate sample problems corresponding to all four complexity indicators
and measure the effect of increasing complexity on CPU time required for solving each
problem. 


[maybe 1 additional paragraph on categorization?]

We follow the work in [ref-De Causmaecker and Berghe] to place our clinician
scheduling problem
in the realm of work for nurse rostering. According to their specification, our
problem includes 
personnel environment constraints targeting the availability of clinicians
([list which constraints...]),
sequences of assignments ([...]), and balance of rosters ([...]) 
which fall under the categories $\alpha : A$, $\alpha : S$ and $\alpha : B$,
respectively. 
For work characteristics, the relevant category for our problem are the range
constraints, $\alpha : R$, 
as we restrict the workload of each clinician between a minimum and maximum
allowed blocks ([constraint...]).
Lastly, for the optimization objective our problem corresponds to the category
$\gamma : P$, as our optimization objective
aim to adhere to clinician preferences. Therefore the appropriate categorization
for our problem is $ASB/R/P$.

Using the classification in [ref-De Causmaecker and Berghe], we compare our clinician
scheduling problem and solution approach with existing literature. Azaiez and Al Sharif [ref]
describe a nurse rostering problem to schedule nurses on a per-day basis, for a total period
of 28 days. In this problem, they are required to assign a minimum number of staff members
per day, unlike in our clinician scheduling problem. Similar to our problem, they include
hard constraints to prevent consecutive assignments, and enforce a minimum and maximum 
number of shifts for each nurse. When formulating the optimization goal, Azaiez and Al Sharif
focus on balancing the workload among nurses, preventing day-night consecutive shifts and avoiding
off-on-off assignments. Their formulation does not incorporate nurse preferences
for time off into the model. 

[...]
