In this paper, we present a simple, yet flexible, integer linear programming
formulation to generate schedules for clinical departments at hospitals.
The challenge in applying ILP to the task of scheduling clinicians lies in the
computational complexity of finding an optimal solution. As the size of the
scheduling problem grows, due to a larger roster of clinicians or more
complicated constraints, the time it takes to generate an optimal schedule may
grow exponentially in the worst case.
Many previous approaches to creating schedules in similar scenarios have avoided
this problem by using heuristics to find an approximately optimal solution 
in a shorter amount of time~\cite{burke_state_2004}.

We presented a formulation that includes both hard constraints to ensure the
schedule satisfies hospital and logistics requirements, and a multi-goal
objective function to satisfy soft constraints (work preferences of clinicians).
Although we restricted our application of the formulation to a set of
constraints for the particular needs of the case study (St.\ Michael's Hospital
Division of Infectious Diseases), our formulation can be adapted to various
clinical departments at different hospitals. The flexibility of our ILP allows
changing the number of services provided in a division, the length of a work
block, clinicians' preference for block to weekend adjacency as well as
clinicians' requests for time off.

When comparing the optimal schedule generated by our tool to the
manually-created schedules at St.\ Michael's Hospital, we found that the ILP
formulation was always able to find an optimal schedule satisfying all required
hard constraints, unlike the manual schedule, which often did not satisfy all
constraints. Moreover, due to the multi-goal objective function in
the ILP, the algorithm was able to fulfill the majority of clinician
preferences and requests, more so than the manually-created schedule. These
observations reinforce the benefits of automated tools when generating schedules
in hospital departments to balance the workload of clinicians and improve the
service provided to patients. The use of automated tools alleviates the time
spent on designing the schedule by hand, and provides clinical departments with
a more fair distribution of work that helps improve the overall satisfaction of
both employees and patients~\cite{silvestro_evaluation_2000}.

In our simulations, we also found that increasing the number of requests per
clinician did not affect the runtime of the algorithm, highlighting the
flexibility of the tool to incorporate clinician preferences. Further, we saw
that the algorithm can accommodate an increase in time-horizon up to four years
with little impact on runtime, suggesting the algorithm can be used generate
schedules far in advance. A key limitation we identified was the sensitivity
of the runtime to larger numbers of services offered by a single department. 
One solution to mitigate the runtime issues created by a
larger number of services would be removing the constraint that prevents
assignment of consecutive blocks, followed by manual readjustment from the
generated schedule.
Overall, our sensitivity analyses using simulated data provided reassurance that
the ILP formulation can be applied to schedule clinicians across real-world
variability between clinical departments.
Next steps include expanding the generalizability of the tool beyond smaller
clinical departments to larger departments within and outside of health-care --
especially those that provide multiple services in parallel for patients and
other clients. As well, additional work can be done to incorporate other
clinician preferences, such as the ability to request time-on slots or preference
for time of year.