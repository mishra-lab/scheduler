In this paper our goal was to develop a rather simple, yet flexible, integer linear programming formulation to generate schedules for clinical consultation (?) departments at hospitals. The difficulty in applying a linear programming approach to this task lies in the fact that ILP is an NP-hard problem, and as such the time to find an optimal solution grows exponentially as we increase the size of the problem. As a result, the vast majority of approaches taken to create schedules for similar scenarios tend to use heuristics in order to find an approximately optimal solution in a shorter time. However, even with the limitations of time complexity we are able to use the tool to generate schedules for a wide variety of simulated departments, supporting upwards of 50 total clinicians. Moreover, our formulation is able to accommodate the requests of clinicians without negatively affecting the runtime. Comparing the generated schedule to a manually created schedule in a real clinical department at a hospital, we see adherence to all required constraints as well as improved fulfillment of clinician requests, which are sometimes overlooked when creating the schedule by hand. The scenarios when the ILP formulation starts to have trouble are ones with multiple services offered within a single clinical department. Once there are more than 10 clinicians in total, providing services in 2 or more divisions, the constraint space becomes very complex, making it difficult for the tool to find an optimal solution in a reasonable amount of time. For these cases, we recommend relaxing the `No Consecutive Blocks` constraint, to simplify the problem and find a solution much faster.