In this paper, we present a simple, yet flexible, integer linear programming
formulation to generate schedules for clinical departments at hospitals.
The challenge in applying ILP to the task of scheduling clinicians lies in the
computational complexity of finding an optimal solution. As the size of the
% JK: Is this a challenge of applying ILP? Or of the problem in general?
%     I got the impression that ILP was a nice way to overcome this problem.
scheduling problem grows, due to a larger roster of clinicians or more
complicated constraints, the time it takes to generate an optimal schedule may
grow exponentially.
%The challenge in applying such an approach to this task lies in the fact that
%ILP is an NP-hard problem. %specify what 'such an approach' and 'this task'
%are... the first part of the sentence was a bit hard to follow. Define what you
%mean by NP-hard problem and cite --> i.e. NP-hard = the following sentence
%about how time to optimal solution grows exponentially? if yes, then instead of
%'As such' - say 'That is,...or "An NP-hard problem means that..."
% JK: How come only "may"? Under what conditions?
As a result, the majority of approaches taken to create schedules for similar
scenarios tend to use heuristics in order to find an approximately optimal
solution in a shorter time~\cite{burke_state_2004}. %cite/reference
%statement (i.e. the 'vast majority'...) I can't remember if the introduction
%details the heuristics --> but if yes, then great. if no, then include that
%review of the prior literature here in discussion. I think as you said - the
%journals we are submitting to seem to have more of those details in the
%introduction.

%double-check all abbreviations have been defined. I would prefer limiting
%abbreviations to no more than 3 at most if possible. We are ok for word count
%in this paper :). 

We presented a formulation that includes both hard constraints to ensure the
schedule satisfies hospital and logistics requirements, and a multi-goal
objective function to satisfy soft constraints (work preferences of clinicians).
%Our formulation includes hard constraints to ensure the schedule satisfies
%hospital and logistics requirements. It also aims to satisfy the work
%preferences of clinicians in a clinical department by optimizing a multi-goal
%objective function. 
Although we restricted our application of the formulation to a set of
constraints for the particular needs of the case study (St.\ Michael's Hospital
Division of Infectious Diseases), our formulation can be adapted to various
clinical departments at different hospitals. The flexibility of our ILP allows
changing the number of services provided in a division, the length of a work
block, clinicians' preference for block to weekend adjacency as well as
clinicians' requests for time off.
% JK: Nice!

When comparing the optimal schedule generated by our tool to the
manually-created schedules at St.\ Michael's Hospital, we found that the ILP
formulation was always able to find an optimal schedule satisfying all required
hard constraints, unlike the manual schedule, which often did not satisfy all
constraints. Moreover, due to the multi-goal objective function in
the ILP, the algorithm was able to fulfill the majority of clinician
preferences and requests, more so than the manually-created schedule. These
observations reinforce the benefits of automated tools when generating schedules
in hospital departments to balance the work-load of clinicians and improve the
service provided to patients. The use of automated tools alleviates the time
spent on designing the schedule by hand, and provides clinical departments with
a more fair distribution of work that helps improve the overall satisfaction of
both employees and patients~\cite{silvestro_evaluation_2000}.  % how the
%finding compares to the wider literature when automated compared to manual?
%What does it mean re: "so what" - human error in generating tools, etc. or
%challenges in heuristics used manually by people? include citations this last
%sentence in particular - what dos it reinforce exactly - what are the benefits?

In our simulations, we also found that increasing the number of requests per
clinician did not affect the runtime of the algorithm, highlighting the
flexibility of the tool to incorporate clinician preferences. Further, we saw
that the algorithm can accommodate an increase in time-horizon up to four years
with little impact on runtime, suggesting the algorithm can be used generate
schedules far in advance. One key limitation we identified was the sensitivity
of the runtime to larger numbers of services offered by a single department.
Such cases are unlikely to be encountered in the real-world because most
clinical departments tend to provide a single service or at most two services by
the same roster of clinicians [{\color{red}SM/Kevin, could you help with
	reference here?}]. One solution to mitigate the runtime issues created by a
larger number of services would be relaxing the constraints that prevent
assignment of consecutive blocks, followed by manual readjustment from the
generated schedule. %what do we mean by certain constraints? specify the other
%constraints?
Overall, our sensitivity analyses using simulated data provided reassurance that
the ILP formulation can be applied to schedule clinicians across real-world
variability between clinical departments. %SM - will look for reference and we
%can ask Kevin Gough too.
Next steps include expanding the generalizability of the tool beyond smaller
clinical departments to larger departments within and outside of health-care --
especially those that provide multiple services in parallel for patients and
other clients. 
